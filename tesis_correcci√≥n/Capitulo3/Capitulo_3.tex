\chapter{Propuesta}\label{cap.Propuesta}
\pagenumbering{arabic} % para empezar la numeración con números
Después de una revisión de información se pudo notar que estos tipos de evaluaciones de manera individual presentan diferentes problemas ya que algunas al tratarse de pruebas que se realizan en laboratorio y en las que los objetivos y tareas se les imponen explícitamente a los participantes, la interacción del usuario se encuentra descontextualizada, influyendo en su forma de resolver problemas.\\
En la evaluación heurística se permite identificar una mayor cantidad de problemas de usabilidad menores, pero una menor cantidad de problemas de usabilidad mayores que otras metodologías como los test de usuarios. Esto significa que esta metodología no puede sustituir a la realización de test de usuarios, ya que resulta menos eficaz en la detección de aquellos problemas de usabilidad que mayor impacto tendrán en el usuario final.\\
Para el card sorting autores coinciden en afirmar que es un método rápido, fiable y barato en que su uso inexperto o inadecuado puede producir resultados erróneos.\\
En el eye tracking el problema es que, aunque el proceso de calibración visual de los participantes previo a la prueba es rápido y sencillo, existe un significativo porcentaje de personas cuyos ojos no pueden calibrarse, lo que encarece aún más este tipo de estudios.\\
Los trabajos que han consistido en evaluar las técnicas de visualización hacen uso de estos métodos de evaluación tablas 2.2 y 2.2 , sin embargo estas evaluaciones presentan ciertos problemas que anteriormente se mencionaron, en este trabajo se propone un marco de trabajo donde estén implicadas más de un tipo de evaluación para un contexto de visualización de información donde esta información será para mostrar el desempeño que este teniendo un usuario dentro de un equipo.

