\chapter{Introducción}\label{cap.introduccion}
\pagenumbering{arabic} % para empezar la numeración con números
Las pruebas y evaluaciones de usabilidad durante el desarrollo sistemas interactivos han ganado amplia aceptación como estrategia para mejorar la calidad del producto. La introducción temprana de las perspectivas de usabilidad en un producto es muy importante para brindar una clara visibilidad de aspectos de calidad, tanto para los desarrolladores como los usuarios de pruebas. Sin embargo, la evaluación y pruebas de usabilidad no es común que se tomen en cuenta como elementos indispensables del proceso de desarrollo de software\cite{florian2013propuesta} .\\
Con el exponencial avance de las tecnologías que prácticamente han invadido progresivamente todos los aspectos de nuestras vidas. En el trabajo, el estudio o el entretenimiento, nuestras actividades cotidianas cada vez se encuentran más automatizadas, con herramientas y sistemas que intentan mejorar nuestra productividad, comodidad y capacidad de acción. En este devenir tecnológico, computadoras y personas están abocados inevitablemente al entendimiento mutuo.\\
Mientras las nuevas tecnologías tanto en hardware como en software contribuyen a un aumento en el volumen y la diversidad de los datos que las organizaciones manejan, la exploración y la visualización de estos datos se tornan cada vez más dificultosas. Por esto, y debido al crecimiento en la cantidad y variedad de usuarios y a la demanda de mayor funcionalidad, es deseable que los sistemas de Visualización de Información (VI) cumplan con propiedades como la extensibilidad, la modificabilidad, la predictibilidad y la posibilidad de estar integrados por componentes intercambiables \cite{martig2003modelo}. \\
Los sistemas interactivos proporcionan información a través de mecanismos visuales que los usuarios utilizan para comprender lo que sucede en el desarrollo de una actividad individual o colaborativa. La información mostrada a los equipos apoya la generación a la toma de decisiones, comunicación, colaboración y coordinación de los integrantes del equipo\cite{garcia2018factores}.Es por esto qué la Visualización de la Información (VI) es un campo de investigación que ha cobrado gran relevancia en el panorama actual.El término Visualización de la Información fue acuñado a finales de los años 80, y hasta ese momento, es considerado como un sector de la disciplina Interacción Humano-Computadora (IHC).\\
El autor Card define la Visualización de la Información como el uso de soporte informático, interactivo, representaciones visuales de datos abstractos para amplificar la cognición\cite{  card1999readings}. El objetivo de las visualizaciones es transformar una estructura en una gráfica, de manera que esta pueda ser visualizada y el usuario pueda interactuar con ella. Algunas de las técnicas de visualización más conocidas tienen un fuerte componente de interacción que ayuda al usuario a explorar de manera rápida los datos \cite{olmeda2014visualizacion}.\\
A medida que el campo de la visualización de la información aumenta las técnicas desarrolladas en laboratorios de investigación están llegando a los usuarios. Los informes de los estudios de usabilidad y experimentos controlados son útiles, pero hay un deseo creciente de métodos alternativos de evaluación, a fin de presentar evidencia medible de los beneficios para la adopción más generalizada de las técnicas visualización \cite{arjona}.\\
Los marcos de trabajo  pueden ayudar a reducir los costos de desarrollo ya que el objetivo de estos es que posibiliten la evaluación de usabilidad automática.

\section{Definición del problema}
El diseño de técnicas de visualización de información son de carácter genérico, estás se pueden utilizar con una amplia variedad de información y utilizadas en muchos dominios de aplicación. Sin embargo, los usuarios podrían estar insatisfechos si perciben que la técnica de visualización de información no ha sido diseñada específicamente para sus necesidades particulares al caer en errores como que estos no comprenda la técnica de visualización  que se ha elegido  y como consecuencia el usuario  no pueda tomar decisiones de acuerdo con la información que se le está presentando\cite{shneiderman}\\
No se trata solamente de divulgar la información en algun tecnica de visualización, sino de profundizar y aprovechar las ventajas que ofrece la percepción humana y la interacción, con el objetivo de transformar realidades complejas y abstractas en realidades simples.

\section{Preguntas de investigación}
\begin{enumerate}
\item ¿Las evaluaciones de usabilidad son suficientes para evaluar las técnicas de VI?
\item ¿Un marco de trabajo permite la evaluación de técnicas de  VI en los sistemas interactivos?
\end{enumerate}

\section{Objetivos}
\subsection{Objetivo General}
Desarrollar un  marco de trabajo para  evaluar técnicas de VI presentes en los sistemas interactivos
\subsection{Objetivos Específicos}
\begin{enumerate}
\item Seleccionar los criterios que permitan evaluar las técnicas de VI.
\item Diseñar el marco de trabajo con los criterios que permitan evaluar  técnicas de VI para representar el desempeño de equipo.
\item Construir un prototipo con implementación de  técnicas de VI.
\item Evaluar  las técnicas de VI en el prototipo propuesto utilizando el marco de trabajo.
\end{enumerate}

\section{Hipótesis}
La construcción de un marco de trabajo para la evaluación de las técnicas de visualización de información ayuda a la toma de decisión de cuál es la técnica de visualización de información conveniente para utlizar en los sistemas interactivos.

\section{Justificación}
El valor de cualquier sistema de información está condicionado por la calidad y cantidad de información contenida, pero al mismo tiempo por su facilidad para encontrar dicha información, cualidad que naturalmente disminuirá conforme aumente el tamaño del sistema \cite{montero}.
\\Paulatinamente, la visualización científica se fue convirtiendo en uno de los métodos más eficaces para la interpretación de dichos datos, sumando cada vez más fanáticos en la comunidad científica. Gracias a ello y a la evolución de la informática, que nos permite contar con computadoras de bajo costo con gran poderío de cómputo gráfico, los espectros de información se han ampliado, extendiéndose hacia otras disciplinas y para casi cualquier dominio \cite{padua}.